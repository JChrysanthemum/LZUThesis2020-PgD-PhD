% !TEX TS-program = xelatex
% !TEX encoding = UTF-8 Unicode

% \documentclass[AutoFakeBold]{LZUThesis}
\documentclass[AutoFakeBold]{LZUThesis2020-PgD&PhD}


\begin{document}
%=====%
%
%封皮页填写内容
%
%=====%

% 标题样式 使用 \title{{}}; 使用时必须保证至少两个外侧括号
%  如: 短标题 \title{{第一行}},
% 	      长标题 \title{{第一行}{第二行}}
%             超长标题\tiitle{{第一行}{...}{第N行}}

\title{{论文}{模板}}



% 标题样式 使用 \entitle{{}}; 使用时必须保证至少两个外侧括号
%  如: 短标题 \entitle{{First row}},
% 	      长标题 \entitle{{First row}{ Second row}}
%             超长标题\entitle{{First row}{...}{ Next N row}}
% 注意:  英文标题多行时 需要在开头加个空格 防止摘要标题处英语单词粘连。
\entitle{{Paper}{ Template}}

\author{匿名}
\major{匿名}
\research{匿名}
\education{学历教育}
\advisor{匿名}
\codvisor{}
\elapse{2020 年 7 月 至 \quad 2021 年 3 月}
\defense{2021 年 5 月}

\maketitle

%======%
%诚信说明页
%授权说明书
%======%
% 如果超出边界,可以调整签字的宽度,现在是50,如果你不用,把下面的注释就好

% 你的签名
\mysignature{
    % \raisebox{-5pt}{
        % \includegraphics[width=40pt]{signature.pdf}
    % }
}
% 你手写的日期
\mytime{
    % \raisebox{-5pt}{
        % \includegraphics[width=40pt]{signature.pdf}
    % }
}
% 老师的手写签名
\supervisorsignature{
    % \raisebox{-5pt}{
        % \includegraphics[width=40pt]{signature.pdf}
    % }
}
% 老师手写的时间
\teachertime{
    % \raisebox{-5pSt}{
        % \includegraphics[width=40pt]{signature.pdf}
    % }
}
% 老师手写的成绩
\recommendedgrade{
    % \raisebox{-5pt}{
        % \includegraphics[width=40pt]{signature.pdf}
    % }
}

\makestatement


\frontmatter



%中文摘要
\ZhAbstract{空白页 空白页 空白页 空白页 空白页 空白页 空白页 空白页 空白页 空白页 空白页 空白页 空白页 空白页 空白页 空白页 空白页 空白页 空白页 空白页 空白页 空白页 空白页 空白页 空白页 空白页 空白页 空白页 空白页 空白页 空白页 空白页 空白页 空白页 空白页 空白页 空白页 空白页 空白页 空白页 空白页 空白页 空白页 空白页 空白页 空白页 空白页 空白页 空白页 空白页 空白页 空白页 空白页 空白页 空白页 空白页 空白页 空白页 空白页 空白页 空白页 空白页 空白页 空白页 空白页 空白页 空白页 空白页 空白页 空白页 空白页 空白页 空白页 空白页 空白页 空白页 空白页 空白页 空白页 空白页 空白页 空白页 空白页 空白页 空白页}{关键词1,关键词2}


%英文摘要
\EnAbstract{example example example example example example example example example example example example example example example example example example example example example example example example example example example example example example example example example example example example example example example example example example example example example example example example example example example example example example example example example example example example example example example example example example example example example example example example example example example example example example example example example example example example example .

\fontspec{Times New Roman} {Times New Roman}}
{key-word-1,key-word-2}

%生成目录
\tableofcontents
\thispagestyle{empty}


%文章主体
\mainmatter

\chapter{第一章}
\section{节}
\subsection{次节}

{\bfseries 编译方式:} XeLaTeX -->BibTeX --> XeLaTeX-->XeLaTeX

原本科模板\cite{partl2016},但是现在需要上标模式\upcite{partl2016}

\begin{table}[H]
    \centering
    \caption{二硫化钼纳米管参数}
    \begin{tabular}{cccccc} % 控制表格的格式,可以是l,c,r
    \toprule
    参数& m & n & \tabincell{c}{太长了\\换行一下\\原子数}  & 内径 & 长度\\
    \midrule
    数值 & 15 & 15  & 2880 & 2.3014nm & 9.95nm \\
    \bottomrule
    \end{tabular}
    \label{tbl_mos2_nanotube}
\end{table}


%论文后部
\backmatter


%=======%
%引入参考文献文件
%=======%
\bibdatabase{bib/database}%bib文件名称 仅修改bib/ 后部分
\printbib
% \nocite{*} %显示数据库中有的,但是正文没有引用的文献



\Achievements
一、发表论文

1.Article here sd

\blank

二、参与课题

1.

2.

3.


\Thanks

这里是致谢页,你可以在这里致谢你的舍友,老师,朋友,或者我。
d



\end{document}
